\documentclass{article}

\title{InsolventByDesign}
\author{}
\date{}

\begin{document}
\maketitle

\section{Introduction}

Cross-chain bridges secure increasing amounts of value while relying on asynchronous consensus and delayed finality. 
Recent auction-based block production mechanisms, such as proposer-builder separation (PBS), transform transaction inclusion into a market-driven process.
This work studies the economic implications of such mechanisms on bridge safety, focusing on the cost of temporarily censoring safety-critical transactions.

Rather than identifying implementation bugs, we analyze whether there exist economic conditions under which censorship-based exploitation becomes rational for an adversary.

\section{Model}

\subsection{Variables}

We define the following variables:

\begin{itemize}
\item $V$ : Total value locked (TVL) secured by the bridge.
\item $\tau$ : Minimum duration for which a safety-critical transaction must be censored.
\item $b(t)$ : Minimum per-slot bribe required to exclude a transaction at time $t$.
\item $C_c$ : Total censorship cost, defined as $C_c = \sum_{t=1}^{\tau} b(t)$.
\item $p(V)$ : Probability that censorship results in successful exploitation.
\item $P(V)$ : Expected attacker profit, defined as $P(V) = p(V)\cdot V - C_c$.
\end{itemize}

\subsection{Attacker Rationality}

We assume an economically rational adversary who initiates an attack if and only if the expected profit is positive.
Formally, an attack is rational when:

\[
P(V) > 0
\]

Equivalently,

\[
p(V)\cdot V > C_c
\]

This inequality defines the fundamental condition under which censorship-based exploitation is economically viable.

\section{Assumptions}

We make the following assumptions throughout the model:

\begin{itemize}
\item Block production is governed by an auction-based mechanism where transaction inclusion is determined by bids.
\item Censorship of a specific transaction requires the attacker to outbid competing transactions for $\tau$ consecutive slots.
\item Builders and proposers are economically rational but not necessarily altruistic.
\item The bridge relies on timely inclusion of safety-critical transactions to prevent exploitation.
\item The attacker requires only temporary censorship sufficient to complete exploitation, not permanent network control.
\item The bridge behavior is fixed over the duration of the attack window.
\end{itemize}

\section{Main Result}

\textbf{Proposition.}
If there exists a value $V^{*}$ such that:

\[
p(V^{*}) \cdot V^{*} > C_c
\]

then censorship-based exploitation is economically rational for an adversary securing value $V^{*}$.

\medskip

This result implies that for fixed censorship cost $C_c$ and non-decreasing $p(V)$, there exists a threshold beyond which bridge security becomes economically unstable under the given assumptions.

\section{Discussion}

The model does not claim inevitability under all protocol designs.
Instead, it establishes a sufficient economic condition under which censorship-based exploitation becomes rational.

Future work includes:
\begin{itemize}
\item Empirical measurement of $b(t)$ using PBS relay data.
\item Analysis of how builder concentration impacts $C_c$.
\item Extension of the model to adaptive bridge mechanisms and inclusion-list constraints.
\end{itemize}

\end{document}
